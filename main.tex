\documentclass{article}
\usepackage[utf8]{inputenc}
\usepackage{fullpage} % Package to use full page
\usepackage{parskip} % Package to tweak paragraph skipping
\usepackage{tikz} % Package for drawing
\usepackage{amsmath}
\usepackage{hyperref}
\usepackage{cancel}
\usepackage{amssymb}
\usepackage{mathtools}
\usepackage{graphicx} % Required to insert images
\graphicspath{{Error_figs/}}
\usepackage{varwidth}
\usepackage{amstext}
\usepackage{amssymb}
\usepackage{float}
\usepackage{array}
\usepackage[section]{placeins}

\title{Radenokovi\'c 2013 \\ Anisotropy analysis of turbulent swirl flow}
\date{}
\begin{document}

\maketitle

\section*{Scope/ Motivation/ Methods}

The paper examines anisotropy characteristics swirl flow in a circular pipe. The experiment is explained in another paper, which is only available in German. Analysis is done on data collected at 12 radial points in the between the center and wall of the pipe. The authors describe that swirl flow can be viewed as 4 distinct regions, vortex core, shear vortex layer, the mean flow\footnotemark, and the wall region. It is noted that parts of swirl flow has negative eddy viscosities, and that swirl flow is "characterized by anisotropy in eddy viscosity". Because of this the "standard two equation model", assuming eddy viscosity as isotropic cannot be applied to swirl flows. 

The paper presents anisotropy analysis with both the anisotropy invariant map (AIM) and the barycentric map (BM). The authors present a thourough formulation of the invariants of the Reynolds stress tensor, and the maps.

\section*{Turb stats, initial figures}
Figure 1 gives mean velocity distributions of both tangential and axial velocities normalized by bulk velocity. Distance is normalized by the radius, as is the case for all the plots in the paper. The tangential velocity component is used to identify the 4 regions in the flow. 

Figure 2 is of radial distributions of the turbulence intensities given by $\sqrt{\Bar{u_i}^2}/U_m$, $U_m$ is again the bulk velocity. The plots show similar profiles in all 3 components, with a range from 8\% to 27\%.

Figure 3 gives the off diagonal components of Reynolds shear stress normalized by $U_m^2$, there is large variation in the profiles, and they collapse on each other at $\approx r/R = 0.65$.

\section*{Detailed descriptions of aniostropy maths}



\footnotetext{It is possible that the mean flow is actually termed "main flow" as it is presented in the introduction, however mean flow is used in section 2, and seems to make more sense.}

\end{document}
